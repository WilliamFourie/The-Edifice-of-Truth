\chapter{Biographies}
\label{Chpt: Biographies} % Hyperref Reference

\section{J\'{a}nos Bolyai (1902 - 1860)} \label{JBolyai}

\section{Ren\'{e} Descartes (1596 - 1650)} \label{RDescartes}

\section{Euclid of Alexandria (c300bce)} \label{Euclid}

\section{Abraham Fraenkel (1891 - 1965)} \label{AFraenkel}

\section{Carl Friedrich Gauss (1777 - 1855)} \label{CGauss}

\section{Gerhard Gentzen (1909 - 1945)} \label{GGentzen} 

\section{Kazimierz Kuratowski (1896 - 1980)} \label{KKuratowski}

\section{Nikolai Lobachevsky (1792 - 1856)} \label{NLobachevsky} And who deserves the credit? And who deserves the blame? Oh, Nikolai Ivanovich Lobachevsky is his name! 

\section{Isaac Newton (1642 - 1726)} \label{INewton} There is a bit of uncertainty over what dates to put for Newton's birth and death. This uncertainty does not state from a lack of knowledge as to the point in history that he was born and died, but rather due to the calendars in use at the time. 

Using the Julian Calendar - the older calendar, in use in Britain, at the time - Newton was born on Christmas Day of 1642, and died on 20 March 1626.

Under the Gregorian calendar - the one we currently use - Newton was born on the fourth of January, 1643, and died on the thirty-first of March, 1627.\footnote{One might wonder why the apparently large discrepancy between his birth dates and his death dates. At the time of Netwon's birth, the Gregorian calendar was ten days ahead of the Julian calendar. At the time of Newton's death, the gap had increased to 11 days. Why, then, does there appear to be almost a year's difference between the death dates? Well, simply put, this was because the new year on the Julian calendar is on the 25th of March, rather than the 1st of January. For more information, see \cite{Christie2015}.}

\section{Bertrand Russell (1872 - 1970)} \label{BRussell} 

\section{Norbert Wiener (1894 - 1964)} \label{NWiener}

\section{Ernst Zermelo (1871 - 1953)} \label{EZermelo}

